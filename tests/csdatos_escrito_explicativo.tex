\documentclass{memoir}
\usepackage[spanish]{babel}
\usepackage{graphicx}

\pagestyle{empty}

\begin{document}
\includegraphics[scale=.5]{ /Users/fernandomurojimenez/Documents/comisiondeservicios/.venv/lib/python3.12/site-packages/csus/logo.pdf}
\begin{vplace}[.5]
\vspace{1cm}
\textbf{Fernando Muro Jiménez}, miembro del proyecto MTM-3000-4000.

\vspace{1cm}

\textbf{QUIERO HACER CONSTAR QUE:}

\vspace{1cm}

Entre los días 6 y 10 de junio de 2022, viajé a Boston, Massachusetts, para asistir al congreso "Mathematical models with industry patterns" organizado por el Massachusetts Institute of Technology. Este evento me permitió profundizar en la investigación relacionada con varios objetivos de mi proyecto, como **O1**, que busca desarrollar nuevos modelos matemáticos para sistemas complejos, y **O8**, que se enfoca en colaborar con socios de la industria para aplicar la investigación matemática a problemas prácticos.

Durante el congreso, también tuve la oportunidad de explorar aplicaciones del álgebra en criptografía (**O2**) y analizar el rol de las matemáticas en la seguridad de redes (**O6**), áreas fundamentales para mi trabajo actual. Estas discusiones fueron especialmente valiosas para mejorar los métodos computacionales en la resolución de ecuaciones algebraicas (**O3**) y para establecer nuevas colaboraciones con expertos del sector. 

\vspace{1cm}

Y para que conste a los efectos oportunos, firmo el presente escrito en Sevilla, a \today.

\vspace{3cm}

\begin{flushright}
    Fdo.: Fernando Muro Jiménez 
\end{flushright}
\end{vplace}
\end{document}